\documentclass[13pt]{article}

\begin{document}

\title{Philosohpy of Artificial Inteligenece}
\author{Dinesh Saini (18111022)}
\date{July 18, 2021}
\maketitle



\section{Summary}

\begin{itemize}

\item The philosophy of artificial intelligence is a branch of the philosophy of technology that explores artificial intelligence and its implications for knowledge and understanding of intelligence, ethics, consciousness, epistemology, and free will.
The technology is concerned with the creation of artificial animals or artificial people (or, at least, artificial creatures; see artificial life) so the discipline is of considerable interest to philosophers.
\item The philosophy of artificial intelligence is a collection of issues primarily concerned with whether or not AI is possible -- with whether or not it is possible to build an intelligent thinking machine.
\item The philosophy of artificial intelligence is also concern of whether humans and other animals are best thought of as machines (computational robots, say) themselves.
\item The most important of the "whether-possible" problems lie at the intersection of theories of the semantic contents of thought and the nature of computation. A second suite of problems surrounds the nature of rationality. A third suite revolves around the seeming “transcendent” reasoning powers of the human mind. These problems derive from Kurt Gödel's famous Incompleteness Theorem. A fourth collection of problems concerns the architecture of an intelligent machine. 
\item AI is achievable, but it will take more than computer science and neuroscience to develop machines that think like people.
\item The focus area of philosophers is judge the potentials of the weak and strong AI. Although, most philosophers hold a grudge against the strong AI and suggest to overthrow it. They accord that strong AI is likely to cause damage unlike weak AI that contributes to the progress of a nation.
\item John McCarthy argues that the philosophy of artificial intelligence is important to consider as it impacts the practice of artificial intelligence.
\item Alan Turing, in his days, had opined that any machine that can imitate a human is intelligent and formulated a test known as The Alan Turing test. In his view, a machine that is able to behave like humans can pass the test. Intelligence is not required.
\item Allen Newell and Herbert A. Simon's physical symbol system hypothesis: "A physical symbol system has the necessary and sufficient means of general intelligent action."
\item Hobbes' mechanism: "For 'reason' ... is nothing but 'reckoning,' that is adding and subtracting, of the consequences of general names agreed upon for the 'marking' and 'signifying' of our thoughts..."
\item The Dartmouth proposal: "Every aspect of learning or any other feature of intelligence can be so precisely described that a machine can be made to simulate it."

\end{itemize}

\end{document}